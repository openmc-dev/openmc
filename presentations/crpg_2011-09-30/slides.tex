\documentclass{beamer}
%\documentclass[handout,t]{beamer}

\batchmode
% \usepackage{pgfpages}
% \pgfpagesuselayout{4 on 1}[letterpaper,landscape,border shrink=5mm]

\usepackage{amsmath,amssymb,enumerate,epsfig,bbm,calc,color,ifthen,capt-of}

\usetheme{Berlin}
\usecolortheme{mit}

\title{The OpenMC Monte Carlo Code}
\author{Paul K. Romano}
\date{September 30, 2011}
\pgfdeclareimage[height=0.5cm]{mit-logo}{mit-logo.pdf}
\logo{\pgfuseimage{mit-logo}\hspace*{0.3cm}}

\AtBeginSection[]
{
  \begin{frame}<beamer>
    \frametitle{Outline}
    \tableofcontents[currentsection]
  \end{frame}
}
\beamerdefaultoverlayspecification{<+->}
% -----------------------------------------------------------------------------
\begin{document}
% -----------------------------------------------------------------------------

\frame{\titlepage}

\section[Outline]{}
\begin{frame}{Outline}
  \tableofcontents
\end{frame}

% -----------------------------------------------------------------------------
\section{Introduction}

\begin{frame}{Background}
  My background:
  \begin{itemize}
    \item<1-> B.S. Nuclear Engineering, RPI (2007)
    \item<1-> M.S. Nuclear Scince and Engineering, MIT (2009)
  \end{itemize}
  Origin of OpenMC:
  \begin{itemize}
    \item<1-> Working on advanced parallelization for Monte Carlo
    \item<1-> Needed testing platform - MC21?
    \item<1-> Ultimately decided to start from scratch
  \end{itemize}
\end{frame}

\begin{frame}{Goals}
  Overall objectives of OpenMC:
  \begin{itemize}
  \item<1-> Fully featured, capable of realistic physics
  \item<1-> Written in a modern programming language (F2003)
  \item<1-> Easy to understand inner workings
  \item<1-> High performance
  \item<1-> Extensible for research purposes
  \item<1-> Open source and freely available
  \end{itemize}
\end{frame}

% -----------------------------------------------------------------------------
\section{Methods and Theory}

\begin{frame}{Geometry}
  \begin{itemize}
  \item All geometry is described as constructive solid geometry (unions and
    intersections of second-order surfaces)
  \item e.g. to construct an annulus
  \end{itemize}
\end{frame}

\begin{frame}{Cross-Sections}
  \begin{itemize}
  \item No need to reinvent the wheel
  \item ACE (A Compact ENDF) Format
    \begin{itemize}
    \item MCNP
    \item SERPENT
    \end{itemize}
  \item Three arrays: NXS, JXS, XSS
  \item Parse arrays into internal derived types
  \end{itemize}
\end{frame}

\begin{frame}{Union Energy Grid}
  \begin{itemize}
  \item Each nuclide has cross-sections tabulated at different energy points
  \item To determine collision type, need $\Sigma$ for each collision
  \end{itemize}
  % Insert picture of pointer method
\end{frame}

\begin{frame}{Tallies}
\end{frame}

\begin{frame}{Parallel Fission Bank}
\end{frame}

% -----------------------------------------------------------------------------
\section{Using OpenMC}

\begin{frame}{Compiling}
\end{frame}

\begin{frame}{XML Input Format}
\end{frame}

\begin{frame}{Running}
\end{frame}

\begin{frame}{Post-run Analysis}
\end{frame}

% -----------------------------------------------------------------------------
\section{Development}

\begin{frame}{Version Control}
\end{frame}

% -----------------------------------------------------------------------------
\section{Conclusions}
\begin{frame}{Questions?}
  References:
  \begin{itemize}
    \item Github \url{http://github.com/paulromano/openmc}.
    \item Documentation \url{http://paulromano.github.com/openmc/}.
  \end{itemize}
\end{frame}
% -----------------------------------------------------------------------------
\end{document}
